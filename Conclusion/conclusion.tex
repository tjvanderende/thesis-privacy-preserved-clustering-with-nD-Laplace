\chapter{Conclusion}
This thesis has explored the application of the kD-Laplace algorithm in training privacy-preserving clustering algorithms on distributed k-dimensional data.
The research was guided by three research questions, which have been addressed as follows: \newline
\textbf{RQ1: How can 2D-Laplace be used to protect the privacy of 2-dimensional data employed for training clustering algorithms?} \newline
Implementing kD-Laplace on n-dimensional data has shown promising results, with the mechanism balancing data utility and privacy protection.
For the Heart and Seeds datasets, the kD-Laplace mechanism scores better in \gls{ami} and adversary advantage for lower privacy budgets (0.1 - 3/5) between kD-Laplace and Piecewise.
The Piecewise mechanism scores better in data utility for higher privacy budgets (5 - 9).

%Between the three variants of kD-Laplace, the variant without any optimization scores the best in data utility, together with the Piecewise mechanism.
%But, if we consider the privacy of the mechanisms, Piecewise scores worse than kd-Laplace.
%Between the clustering algorithms, K-means scores the highest \gls{ami} in data utility for both mechanisms.

\textbf{RQ2: How can 3D-Laplace be used to protect the privacy of 3-dimensional data employed for training clustering algorithms?} \newline
\todo[inline]{Results}
\textbf{RQ3: How can nD-Laplace be used to protect the privacy of n-dimensional data employed for training clustering algorithms?} \newline
This research question consists of three hypotheses, which are addressed as follows: \newline
\begin{itemize}
	\item H1: Adding remapping based on density improves utility without sacrificing privacy:
	      \todo[inline]{In progress}

	\item H2: The privacy leakage (adversary advantage) and utility increases for more dimensions.
	      \todo[inline]{In progress}
	      %We evaluated only the kd-Laplace/grid/optimal variant of the kD-Laplace mechanism.

	      %The heart-dataset's \gls{tpr} increases with more dimensions for the lower privacy budgets.
	      %The seeds dataset shows a different trend, with the \gls{tpr} decreasing slightly with more dimensions.
	      But especially for the higher privacy budgets.
	\item H3: The shape of the data negatively impacts the kd-Laplace mechanism in terms of privacy and utility.
	      The kD-Laplace mechanism scores worse for \gls{ami} for the shape datasets in comparison to the real-world datasets (Heart and Seeds).
	      Piecewise, conversely, scored more consistently and better in \gls{ami} for the shape datasets.
	      Regarding the adversary advantage, kD-Laplace scores higher than Piecewise.
	      Upon closer examination, we observe that the True Positive Rate (TPR) is also higher for kD-Laplace. For both the Circle and Line datasets,
	      kD-Laplace significantly outperforms Piecewise.
	      Therefore, the shape of the data negatively impacts the kd-Laplace mechanism in terms of privacy and utility.
	      %We evaluated three shapes: Circle, line, and skewed, and compared the kD-Laplace/optimal/grid mechanism to Piecewise.
	      %For the circle, the Piecewise mechanism's utility is best, but privacy-wise worse.
	      %The kd-Laplace mechanism scores better in utility and privacy in the line dataset.
	      %Also, the kd-Laplace mechanism scores better in utility for the skewed dataset.
	      %However, the kD-Laplace shows a higher \gls{tpr} than the Piecewise mechanism for the skewed dataset.
\end{itemize}

\todo[inline]{Still in progress}

Our investigation into the utility and privacy differences among the three kd-Laplace variants and the Piecewise mechanism has revealed that the kd-Laplace mechanism can effectively protect the privacy of n-dimensional data.
The kd-Laplace mechanism generally scores better in data utility and privacy than the Piecewise mechanism for lower privacy budgets (0.1 - 3/5).
The findings of this research have significant implications for the field of privacy-preserving data analysis.
This research approach presents a fresh take on a solution for protecting the privacy of n-dimensional data used in clustering algorithms.
Also, this thesis presents a new take on evaluating privacy mechanisms using real-world data and attacks.

However, further research is needed to address the limitations identified in this study.
In particular, the impact of the shape of the data on the mechanism's effectiveness, also for dimensions higher than 2-dimensions.
In addition, it is also worth exploring the impact of the number of dimensions on the privacy and utility of the mechanism.

In conclusion, this thesis has contributed to understanding how the kd-Laplace algorithm can be applied in training privacy-preserving clustering algorithms on distributed k-dimensional data.
The findings provide a foundation for future research in this area, with the potential to advance the field of privacy-preserving data analysis.




