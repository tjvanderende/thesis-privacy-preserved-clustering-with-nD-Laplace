\newglossaryentry{rig}
{
	name=Rand Index,
	description={
			Compares the similarity between two clusters by comparing all pairs.
			It can therefore be used to measure the performance between two clustering algorithms \citep{hubert_comparing_1985}. }
}
\newglossaryentry{arig}
{
	name=Adjusted Rand Index,
	description={
			The Rand Index is improved and adjusted for chance \citep{hubert_comparing_1985}.
			This algorithm takes also into consideration the number of clusters and can be used to also compare different cluster algorithms \citep{gates_impact_2017}.}
}
\newglossaryentry{mig}{
	name=Mutual Information,
	description={This metric can be used to explain the amount of information about a random variable if compared to another random variable.
			Therefore, it can also be used to compare two cluster similarities.}
}
\newglossaryentry{amig}{
	name=Adjusted Mutual Information,
	description={Comparable with \gls{arig} this algorithm is modified to account to chance.
			This means it accounts for a higher MI for a higher amount of clusters between two cluster algorithms. Therefore, the calculations are strongly influenced by that of \gls{arig} \citep{vinh_information_2009}. }
}


\newglossaryentry{nmig}{
	name=Normalized Mutual Information,
	description={The normalized version is a scaled version of \gls{mig} to always be a value between 0 (no correlation) and 1 (perfect correlation).
			This version of \gls{mig} is not adjusted and therefore highly influenced by cluster amount \citep{vinh_information_2009}. So it suffers the same issue as with \gls{mig}.}
}

\newglossaryentry{bvg}{
	name=Bit Vector,
	description={List or array to store several bits.}
}

\newglossaryentry{aeeg}{
	name=Average Estimation Error,
	description={This is the difference between an estimated value and the real value.}
}

\newglossaryentry{chig}{
	name=Calinski-Harabasz Index,
	description={This is a way to measure the similarity of clusters \citep{calinski_dendrite_1974}.
			It tells how well the clusters are separated from each other and how well the points are grouped.}
}
