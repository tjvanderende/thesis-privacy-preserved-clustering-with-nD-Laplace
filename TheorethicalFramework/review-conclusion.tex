\subsection{Evaluation}
It is important to compare our kD-Laplace method with other studies for this research.
Studies that are similar to ours utilize \gls{ldp} for clustering and add noise to cluster centroids \citep{xia_distributed_2020, yuan_privacypreserving_2021, 9679364} or require modifications of the K-Means algorithm \citep{sun_distributed_2019}.

As explained in Section \ref{theory:literature-review:dp-clustering}, we will use input-perturbation.
So, our methodology remains independent of the specific clustering algorithm chosen.
Based on the literature study, two privacy mechanisms are comparable for this approach: Harmony \citep{nguyen_collecting_2016} and Piecewise \citep{wang_collecting_2019}.

Both mechanisms focus on solving the problem in Duchi et al.'s paper.
However, the Harmony mechanism focuses on solving a paper from 2013 \citep{duchi_privacy_2013} and the Piecewise mechanism on Duchi et al.'s paper from 2017 \citep{duchi_minimax_2017}.
The Piecewise mechanism is newer, and given the similarities, Wang et al.'s paper appears much like a more recent version of Nguyen et al.'s paper.
Therefore, we only selected Piecewise to compare with our mechanism.

\subsubsection{Duchi et al.'s mechanism}
The Piecewise mechanism is based on Duchi et al.'s mechanism for one-dimensional data.
Therefore, we start by explaining this mechanism first.
Duchi et al.'s mechanism is a relatively simple method based on the Bernoulli distribution \citep{duchi_minimax_2017}.
This distribution yields either a 0 (negative) or a 1 (false) based on specified probabilities \footnote{https://mathworld.wolfram.com/BernoulliDistribution.html}.
The mechanism works on a domain tuple [-1, 1] and returns -1 or 1.
So, the \gls{pdf} of this function is as follows \footnote{https://mathworld.wolfram.com/BernoulliDistribution.html}:
\begin{equation}[H]
  P(n) =
  \begin{cases}
    1 - p & \text{if } n = 0 \\
    p     & \text{if } n = 1
  \end{cases}
\end{equation}
The probabilities are calculated based on the input value and can then be used to estimate a mean value.
Compared to Laplace, Duchi et al. solution performs better in variance for epsilons smaller than 2 \citep{wang_collecting_2019}.
However, it performs worse when the epsilon value is higher because the algorithm does not consider the privacy budget for values that are 0.
The multidimensional variant looks like the one-dimensional version but samples each data point's noise independently.

\subsubsection*{Piecewise mechanism} \label{theory:piecewise}
The authors aim to create a method that combines the advantages of the Laplace mechanism and Duchi et al.'s methods \citep{wang_collecting_2019}.
The goal is to reduce the variance for a broader range of privacy budgets.
Like the mechanism above, the Piecewise mechanism only accepts data from -1 to 1.
We first explain the one-dimensional variant and then the multidimensional variant. \newpage

The mechanism takes an input of the range $[-1, 1]$ and returns a perturbed value in the same range.
It uses a piecewise function (hence the name), which is a function that is defined by multiple sub-functions.
The Piecewise mechanism starts with randomly selecting a value $x$ between 0 and 1 and a value $t \in R^1$.
Then, based on this value the mechanism re-assigns a random value to $x$, but with a modified domain (based on the initial value of $x$) \citep{wang_collecting_2019}:
\begin{equation}
  t = \begin{cases}
    [l(t), r(t)] & \text{if } x < \frac{e^{\epsilon/2}}{e^{\epsilon/2}+1} \\
    [-C, l(t) \cup r(t), C]
  \end{cases}
\end{equation}
%According to this formula, the outcome domain shifts depending on the value of $t$.
\begin{enumerate}
  \item Calculated using the privacy budget $\epsilon$, the $C$ determines the amount of noise by bounding the lower/ upper limit of the domain.
  \item $l(t)$ and $r(t)$ are the left and right boundaries of the domain. These are calculated based on the value of $t$ and the privacy budget.
\end{enumerate}
Depending on the value of $t$, the domain is shifted to the left or right on a domain of three "pieces": $[-1, 0, 1]$ \citep{wang_collecting_2019}.
According to Wang et al., when $t = 0$, the probability of $t \in [l(t), r(t)]$ is higher.
In this case, the mechanism generates a perturbed value close to $0$ (depending on the privacy budget).
This phenomenon is the same for $t = 1$ and $t = -1$. \newline
%Where $C$ is defined according to the following mathematical formula:
%\begin{equation}
%  C = \frac{exp(\epsilon/2)+1}{exp(\epsilon/2) - 1} \\
%  \label{fig:piecewise-C}
%\end{equation}
%The noise is sampled from this distribution by conditionally executing one of two algorithms (see for reference Wang et al.'s paper) \citep{wang_collecting_2019}.

%Due to the symmetric nature of their mechanism's \gls{pdf}, the authors can handle the value of 0.
%Imagine it as a histogram with the distribution centered around 0. \todo[inline]{mathematical formula?}
%The histogram consists of three "parts" on the left, right, and center (0).
%This approach allows for determining the probability of 0 based on a certain likelihood, unlike Duchi et al.'s solution.

The authors generate a randomly sampled set of values $k$ for the multidimensional approach.
Next, for each value from $k$, the value is sampled using the one-dimensional Piecewise mechanism.
A critical factor in the overall calculation is the scale factor $C_d$, which, in addition to $C$ (referring to Equation \ref{fig:piecewise-C}), also considers the number of dimensions $d$.
Finally, the authors also provide an extension to support categorical data.


