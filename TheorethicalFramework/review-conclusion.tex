\subsection{Evaluation}
It is important to compare our nD-Laplace method with other studies for this research.
Studies that are similar to ours utilize \gls{ldp} for clustering and add noise to cluster centroids \citep{xia_distributed_2020, yuan_privacypreserving_2021, 9679364} or require modifications of the K-Means algorithm \citep{sun_distributed_2019}.

As explained in Section \ref{theory:literature-review:dp-clustering}, we will use input-perturbation.
So, our methodology remains independent of the specific clustering algorithm chosen.
Based on the literature study, two privacy mechanisms are comparable for this approach: Harmony \citep{nguyen_collecting_2016} and Piecewise \citep{wang_collecting_2019}.

Both mechanisms focus on solving the problem in Duchi et al.'s paper.
However, the Harmony mechanism focuses on solving a paper from 2013 \citep{duchi_privacy_2013} and the Piecewise mechanism on Duchi et al.'s paper from 2017 \citep{duchi_minimax_2017}.
The Piecewise mechanism is newer, and given the similarities, Wang et al.'s paper appears much like a more recent version of Nguyen et al.'s paper.
Therefore, we only selected Piecewise to compare with our mechanism.

\subsubsection{Duchi et al.'s mechanism}
The Piecewise mechanism is based on Duchi et al.'s mechanism for one-dimensional data.
They describe how, in different types of estimation problems, the ideal estimation rate changes based on the privacy level and other problem-related factors \citep{duchi_minimax_2017}.
Their own solution is centered on providing a \gls{ldp} method utilizing the Bernoulli distribution \citep{duchi_minimax_2017}. 
In this distribution, outcomes can be either 0 (representing negative) or 1 (representing positive) based on specified probabilities \footnote{https://mathworld.wolfram.com/BernoulliDistribution.html}. \newline
The proposed mechanism can perturb a tuple $t_i$ from the range [-1, 1] to a tuple $t_i'$ according to the following probabilities \citep{wang_collecting_2019}:
\begin{equation}
    Pr [t_i' = x \ | \ t_i ] = 
    \begin{cases}
        \frac{e^\epsilon - 1}{2 \cdot e^\epsilon + 2} \cdot t_i + \frac{1}{2}, & \quad \text{if} \ x = \frac{e^\epsilon + 1}{e^\epsilon - 1} \\
        - \frac{e^\epsilon - 1}{2 \cdot e^\epsilon + 2} \cdot t_i + \frac{1}{2}, & \quad \text{if} \ x = - \frac{e^\epsilon + 1}{e^\epsilon - 1}
    \end{cases}
\end{equation}
In this context, $\epsilon$ (privacy budget) serves the same purpose as it does for other privacy mechanisms.
Duchi et al. demonstrates, through their proof of this formula, that the average value $t_i'$ closely aligns with the original value $t_i$ \citep{duchi_minimax_2017, wang_collecting_2019}.
When compared to the Laplace mechanism, Duchi et al.'s solution demonstrates superior performance in terms of variance when the privacy budget is less than 2 \citep{wang_collecting_2019}. 
However, its performance declines with higher privacy budgets because the algorithm does not account for the privacy budget when values are set to 0.

\subsubsection*{Piecewise mechanism} \label{theory:piecewise}
The authors aim to create a method that combines the advantages of the Laplace mechanism and Duchi et al.'s methods \citep{wang_collecting_2019}.
The goal is to reduce the variance for a broader range of privacy budgets.
We first explain the one-dimensional variant and then the multidimensional variant. \newpage

As with Duchi et al's mechanism Piecewise takes an input of  $t_i \in [-1, 1]$ and returns a perturbed value in the range $t_i' \in [-C, C]$, where C \citep{wang_collecting_2019}:
\begin{equation}
    C = \frac{e^{\epsilon/2} + 1}{e^{\epsilon/2} - 1}
\end{equation}
The PDF of $t_i'$  is defined as a piecewise function (hence the name) \citep{wang_collecting_2019}:
\begin{equation}
    pdf [t_i' = x \ | \ t_i ] = 
    \begin{cases}
        p & \text{if} \ x \in [l(t_i), r(t_i)] \\ 
        \frac{p}{e^\epsilon} & \text{if} \ x \in [-C, l(t_i) \cup r(t_i), C]
    \end{cases}
    \label{eq:piecewise-domain}
\end{equation}
Where the $p$ , $l(t_i)$ and $r(t_i)$ are defined as \citep{wang_collecting_2019}:
\begin{align}
    \label{piecewise-p}
    & p = \frac{e^\epsilon - e^{\epsilon/2}}{2 \cdot e^{\epsilon/2} + 2} \\ 
    \label{piecewise-l-i}
    & l(t_i) = \frac{C + 1}{2} \cdot t_i - \frac{C - 1}{2} \\
    \label{piecewise-r-i}
    & r(t_i) = l(t_i) + C - 1
\end{align}
%According to this formula, the outcome domain shifts depending on the value of $t$.
Based on the privacy budget $\epsilon$, the $C$ determines the amount of noise by bounding the lower/ upper limit of the domain.
In cases where $t_i = 0$ there is a higher likelihood that the perturbed value $t_i'$ will be closer to $[l(t_i), r(t_i)]$ rather than moving towards $-C$ or $C$."
Similar adjustments occur when $t_i = 1$ or $t_i = -1$ but in those instances, the perturbed values tend to move towards $C$ or $-C$ respectively. 
This results in the mechanism effectively shifting the value of $t_i$ between three distinct values ("pieces") [$-1, 0, 1$] based on the provided probability density function (PDF) \citep{wang_collecting_2019}.

%Where $C$ is defined according to the following mathematical formula:
%\begin{equation}
%  C = \frac{exp(\epsilon/2)+1}{exp(\epsilon/2) - 1} \\
%  \label{fig:piecewise-C}
%\end{equation}
%The noise is sampled from this distribution by conditionally executing one of two algorithms (see for reference Wang et al.'s paper) \citep{wang_collecting_2019}.

%Due to the symmetric nature of their mechanism's \gls{pdf}, the authors can handle the value of 0.
%Imagine it as a histogram with the distribution centered around 0. \todo[inline]{mathematical formula?}
%The histogram consists of three "parts" on the left, right, and center (0).
%This approach allows for determining the probability of 0 based on a certain likelihood, unlike Duchi et al.'s solution.
For multidimensional data, the authors use the one-dimensional version multiple times.
However, they compensate for the number of dimensions by generating a $k$ based on the following formula \citep{wang_collecting_2019}:
\begin{equation}
  k = max (1, min (d, \lfloor \frac{\epsilon}{2.5} \rfloor))
\end{equation}
Here, $d$ is the number of dimensions, and $k$ is used to sample a $d$ number of random samples.
Finally, the Piecewise mechanism (see Equation \ref{eq:piecewise-domain}) is executed for each sample.

\todo[inline]{Check}