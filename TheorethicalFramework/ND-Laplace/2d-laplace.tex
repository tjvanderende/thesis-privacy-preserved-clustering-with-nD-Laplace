%
\newglossaryentry{X}
{
  type=genericmath,
  name={$\ensuremath{X} $},
  description={Set of locations for a user. ($R^2$)},
}
\newglossaryentry{Z}
{
  type=genericmath,
  name={$\ensuremath{Z} $},
  description={For every $x \in X$ a perturbed location $z \in Z$ is reported.},
}
\newglossaryentry{privacy level}
{
  type=genericmath,
  name={$\ensuremath{l} $},
  description={Privacy level},
}
\newglossaryentry{radius}
{
  type=genericmath,
  name={$\ensuremath{r} $},
  description={Radius},
}
\newglossaryentry{Epsilon}
{
  type=genericmath,
  name={$\ensuremath{\epsilon} $},
  description={Defined as $\epsilon = l/r$},
}
\newglossaryentry{theta}
{
  type=genericmath,
  name={$\ensuremath{\theta} $},
  description={Angle},
}

\newpage
\section{2D-Laplace}
The idea of \gls{gi} was introduced to address the issue of privacy and location data \citep{DBLP:journals/corr/abs-1212-1984} (See Equation \ref{algo:2d-geo-indistinguishability}).
It offers an alternative approach for achieving (local) differential privacy for geographical data (latitude/longitude).
The mechanism achieves this by locally adding noise to the location before sending it to a location-based system (LBS).
This section starts with an introduction to mathematics, and for each of the different subsections, we visualize and explain open challenges and theoretic for applying them for clustering.
%\glsaddall
%\leading{10pt}
%\printglossary[type=genericmath, nonumberlist]
%The other symbols can be found in section \ref{section:dp}.
\subsection{Planar and polar Laplace}
In Section \ref{theory:geo-indistinguishability}, we provided an explanation of the concept of \gls{gi}. 
Additionally, we introduced the notion that when a point $x_0 \in X$ serves as the center of a range $r$, a point $z \in Z$ is generated using a noise function \citep{DBLP:journals/corr/abs-1212-1984}. The objective here is to ensure that when the actual locations are $x_0$ and $x_0'$, their divergence is limited to at most $e^{-\epsilon \cdot d(x_0, x_0')}$. This characteristic aligns with the Laplace distribution, which the "planar Laplace" mechanism leverages.
Furthermore, it is worth noting that the "planar Laplace" mechanism is an adaptation of the Laplace distribution designed to accommodate distances in a 2-dimensional space \citep{DBLP:journals/corr/abs-1212-1984}. For clarity, we will refer to this method as "2D-Laplace" from this point forward.

The distance method $d(\cdot, \cdot)$ is a method to calculate the Euclidean distance between two points.
Recalling the definition of Laplace, this method $|x-x'|$ is replaced by the distance metric.
Given the actual location $x_0 \in X$, the \gls{pdf} of the noise mechanism on any other point $z \in Z$ is provided as \citep{DBLP:journals/corr/abs-1212-1984}: 
\begin{equation}
  D_\epsilon(x_0)(z) = \frac{\epsilon^2}{2 \cdot \pi}e^{-\epsilon \cdot d(x_0, z)}
  \label{eq:polar-laplace-pdf}
\end{equation}
The method works for Cartesian coordinates but was modified to support polar coordinates by including $\theta$.
So each polar coordinate is reflected as $(r, \theta)$, where $r = d(x_0, z)$ around point $x_0$.
This idea is visualized in the following figure:
\begin{figure}[H]
  \includesvg[scale=1]{TheorethicalFramework/ND-Laplace/Images/polar-laplace.svg}
  \centering
  \caption{Representation of the generated $(r,\theta)$ and original point $x_0$.}
  \label{figure:parea}
\end{figure}
Next we show in detail, how to generate the polar coordinates $(r, \theta)$. \newline
\textbf{Calculating $r$:}
The $r$ is randomly selected based on a distribution $R$ \citep{DBLP:journals/corr/abs-1212-1984}: 
\begin{equation}
    D_{\epsilon, R}(r) = \int^{2\cdot \pi}_0 \ D_\epsilon(r, \theta) \ d\theta = \epsilon^2 \ r \epsilon^{-\epsilon \cdot r}
    \label{2d:generate-r}
\end{equation}
Where the $r$ is selected randomly on the area of the circle. 
Next, it can be randomly drawn by inverting the \gls{cdf} for the Laplace distribution \citep{DBLP:journals/corr/abs-1212-1984}:
\begin{equation}
  C{_\epsilon}{^{-1}}(p) = - \frac{1}{\epsilon}(W_-1 (\frac{p - 1}{e}) + 1)
  \label{eq:lambert_w_1}
\end{equation}
For this equation, the Lambert $W$ function is used. This function consists of two different branches \citep{corless_lambertw_1996}. This means the value of $W_0(x)$ is always positive, while $W_{-1}(x)$ is always negative. The Lambert w function (also called the product logarithm) is defined as $W(x)e^{W(x)} = x$ \citep{lehtonen_lambert_2016}.
The purpose of the Lambert W function is to invert the \gls{cdf} of the Laplace distribution to generate random noise for one of the coordinates ($r$) using the random value of $p$. It draws the $r$ which will be bounded by the $W_{-1}$, this is very useful for drawing the random planar noise \citep{corless_lambertw_1996}.

\textbf{Calculating $\theta$:}
The other variable ($\theta$) is defined in a similar way \citep{DBLP:journals/corr/abs-1212-1984}: 
\begin{equation}
    D_{\epsilon, \Theta}(\theta) = \int^\infty_0 D_\epsilon(r, \theta) \ dr = \frac{1}{2 \cdot \pi}
    \label{2d:generate-theta}
\end{equation}
To visualize the data, it is necessary to convert the polar coordinates for $(r, \theta)$ to Cartesian coordinates $z = (x, y)$.
This conversion is described as step 4 of the planar Laplace algorithm \citep{DBLP:journals/corr/abs-1212-1984} and visualized using figure \ref{figure:geo}.
\begin{figure}[h]
    \centering
  \includesvg[width=0.8\textwidth]{TheorethicalFramework/ND-Laplace/Images/polar-laplace-to-planar.svg}
  \centering
  \caption{Representation of converting the polar coordinate $(r, \theta)$ to a perturbed point $z = (x, y)$.}
  \label{figure:geo}
\end{figure}

\newpage
\subsection{Truncation} \label{theory:truncation}
After adding the noise to the data, it cannot be ensured the data is within the original domain (figure \ref{figure:truncation-2d}).
If this is not the case, the data is easily distinguished by an unwanted adversary \citep{DBLP:journals/corr/abs-1212-1984,9646489}.
The truncation is an essential part of the mechanism to ensure the data is contained within the domain of the original data $X$. The following example shows two different original points $x_0$ and $x_0'$, where $z$ and $z'$ are being remapped to be within the domain:
%We assume a user has a set of data points with a range of [-1, 1].
\begin{figure}[H]
\centering
  \includesvg[width=1\textwidth]{TheorethicalFramework/ND-Laplace/Images/remapping.svg}
  \caption{Representation of truncation of data points for 2-dimensional Laplace mechanism.}
  \label{figure:truncation-2d}
\end{figure}
%A solution was described by Andres et al. in step 5 of the Laplacian mechanism for 2D space \citep{DBLP:journals/corr/abs-1212-1984}.
%A viable solution is to create a grid around the diameter of the set of points $X = R^2$ that belong to the user \citep{DBLP:journals/corr/abs-1212-1984}.
This approach was introduced by Andres et al. to remap a perturbed point $z$ to the closest point in $G$ \citep{DBLP:journals/corr/abs-1212-1984}.
Here, $G$ is a grid with sides $u$ and $v$, such that $u \leq v$.
%Although this approach remaps data within the original domain of $X$, it is not guaranteed it preserves \gls{gi} anymore.
Let the below equation be the collection of probabilities for a point $z$ being remapped to a closest point in $G$:
\begin{equation}
  R(z) = \{ \ y \in R^2 \ | \ \forall z' \in G \cdot d(y, z') \leq d(y, z') \ \}
  \label{eq:grid-probability}
\end{equation}
The original \gls{gi} definition contains $K$, which is the probability of $z$ being reported as $x_0$ (See Equation: \ref{theory:geo-indistinguishability}).
However, this probability is no longer guaranteed because $z$ can also be part of $G$ \citep{DBLP:journals/corr/abs-1212-1984}.
Hence the probability is now $R(z) = G \cap A$. Where $A$ is a set of acceptable datapoints.\newline
So, $R(z)$ has a different shape depending on the distance $x_0$ and $z$ (ergo, it depends on the grid unit $v$ or $u$). This is due the step units of $G$ stay the same, while the distance $r$ grows \citep{DBLP:journals/corr/abs-1212-1984}.

To overcome this issue, Andres et al. propose a way of calculating $\epsilon'$, depending on the step-unit $u$. \citep{DBLP:journals/corr/abs-1212-1984}.
They proved this and provided theorem 4.1 \citep{DBLP:journals/corr/abs-1212-1984}:
\begin{theorem}[Discretization 2D-Laplace]
  Assume $r_{max} < \frac{u}{\delta_{\theta}}$, and let $q = \frac{u}{r_{max}}\delta_{\theta}$. \\ Let $\epsilon$, $\epsilon' \in R^+$ such that \\
  $\epsilon' + \frac{1}{u}ln \frac{q + 2 e^{\epsilon'u}}{q - 2 e^{\epsilon'u}} \leq \epsilon$ \\
Then $K_{\epsilon'}$ provides $\epsilon$-geo-indistinguishability within the range of $r_{max}$. \\ 
  Namely, if ${d(x_0, z), d(x'_0, z) \leq r_{max}}$ then: \\
  $K_{\epsilon'}(x_0)(z) \leq e^{\epsilon d(x_0, x'_{0})} K_{\epsilon'}(x'_{0})(z)$.
  \label{theorem:discretization}
\end{theorem}
Where $x_0$ and $x_0'$ are two different original points for which the noise is generated. 
Here, $\delta_{\theta}$ is the machine's precision, which is the hardware precision of the GPS-location in the context of geographical data. We will omit this in our research, but still provide the full theorem nonetheless.
The theorem states that $\epsilon'$ is the additional noise needed to satisfy \gls{gi} with the introduction of discretization.
%Then, the final step is truncation based, which is based on the discretization \citep{DBLP:journals/corr/abs-1212-1984}.
It is sufficient to take $r_{max}$ as $diam(A)$, which is the diameter of the set of points $A$ if it satisfies theorem \ref{theorem:discretization} \citep{DBLP:journals/corr/abs-1212-1984}:
So, $r_{max}$ is the maximum distance between points in $A$, which is the area where geo-indistinguishability can be guaranteed \citep{9646489}.
%This idea was later improved by Chatzikokolakis et al., introducing an optimized way of remapping \citep{chatzikokolakis_efficient_2017}.
%The algorithm uses the Bayesian rule to minimize the loss of utility while remapping the data.
%Instead of remapping to the closest point, it remaps to a location where the loss is minimal.
%To decrease the performance impact of this algorithm, it is possible only to consider a specific region around the perturbed point $z$.
%The disadvantage of this method is the need for a prior set of data points to calculate the optimal remapping.
%It does not work for new users and extends the training period.
\subsection{Final mechanism}
Finally, we provide as means of a summary the final algorithm for the Laplace mechanism for 2D space
\begin{algorithm}[H]
  \caption{Full mechanism for perturbing training data for planar/2D-Laplace \citep{DBLP:journals/corr/abs-1212-1984}}\label{alg:rq1}
  \begin{algorithmic}
    \Require $X$, $\epsilon$, $u$, $v$ 
    \Ensure $Z$ 
    %\State $r = \frac{\sigma}{2}$ \Comment formula 4.1
    %\State $\epsilon = \frac{l}{r}$ \Comment Calculating privacy budget \citep{DBLP:journals/corr/abs-1212-1984}
    \State Generate $\epsilon'$         \Comment{Theorem: \ref{theorem:discretization}}
    \State $Z \gets \varnothing$
    \State Generate G from given sides $u, v$.
    \State Generate $A$ with acceptable points around $x_0$ 
    \For{$x_i \in X$}
    \State Generate $\theta$       \Comment Equation: \ref{2d:generate-theta}.
    \State Generate the radius $r$ around $x_0$ \Comment Equation: \ref{eq:lambert_w_1}.
    %\State $z_i \gets T(x_{min}, x_{max}, point_i, z_i)$ \Comment algorithm 1.
    \State Convert polar coordinate $(r,\theta)$ to Cartesian coordinates $z = (x, y)$ \Comment Figure: \ref{figure:geo}.
    \If{z $\notin A$}
        \State Remap $z$ to closest point in $A \cap G$. 
    \EndIf
    \State Add $z$ to perturbed set $Z$.
    \EndFor
    \State \Return Z
  \end{algorithmic}
  \label{alg:2d-laplace}
\end{algorithm}
\newpage