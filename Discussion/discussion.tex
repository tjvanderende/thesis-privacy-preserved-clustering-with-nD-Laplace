\chapter{Discussion}
For each result, we include an interpretation and discussion. This section summarizes these findings and suggests future research directions.
Our study involved a series of experiments to assess the utility and privacy attributes of the nD-Laplace mechanism compared to the Piecewise mechanism. We employed five datasets in our analysis, two of which are real-world datasets: seeds-dataset and heart-dataset. The remaining three, Circle, Line, and Skewed, are synthetic datasets crafted to evaluate the performance of nD-Laplace.

Our initial focus was on the utility of three clustering algorithms: K-Means, \gls{ag}, and \gls{optics}. We then delved deeper to discern the disparities between the Piecewise mechanism and nD-Laplace, examining both in terms of utility—using internal (\gls{sc}) and external (\gls{ami}) validation—and privacy, with an emphasis on metrics like membership advantage and \gls{tpr}. A subsequent experiment incorporated the variants of nD-Laplace, using the same evaluation metrics.

To ensure reproducibility, our research was conducted using a Jupyter notebook. We also used a version control system (git) to store all datasets, safeguarding against any external alterations that might skew the results. For robustness, each experiment was repeated 10 times, and the average result was considered. 

Our findings underline the significant influence of the privacy budget on the utility and privacy of nD-Laplace. As the privacy budget increases, utility improves, but at the cost of reduced privacy. This observation is consistent with the existing literature \citep{sun_distributed_2019, xia_distributed_2020, 9679364}. \newline

In comparing nD-Laplace with the Piecewise mechanism, it was clear that nD-Laplace excelled with the heart data set, seeds data set and skewed data set, particularly when evaluated using K-Means and \gls{ag} clustering \gls{ami} scores. However, its performance was subpar for the line-dataset and circle dataset. Also, \gls{optics} yielded a notably low \gls{ami} score.

Dimensionality had a marginal effect on the utility and privacy for the seeds and heart-dataset. The most significant outcome was observed for the top two dimensions, which demonstrated a minor increase in \gls{ami}, suggesting enhanced utility with increased dimensions.  This might be due to our observation for the hypersphere volume (see Section: \ref{theory:privacy-utility-nd}), where we stated the possibility of an increased utility for 7 > dimensions. This would need further investigation to know for sure.
In addition, the enhancement in utility did not translate into a corresponding increase in privacy. 

In our concluding experiments, we delved into the details of the nD-Laplace mechanism by examining its variants: grid-nD-Laplace and density-nD-Laplace. The primary objective of these variants is to ensure that the data processed by nD-Laplace remain confined within its original domain.

Drawing from existing literature, we recognized the criticality of employing a remapping or truncation method. Without such a method, data with a low epsilon value risk being plotted outside its original domain. This deviation can potentially provide attackers with an advantage, enabling them to infer whether the data is part (member) of the original dataset.

Our empirical findings agreed on these findings. Specifically, when operating with a limited privacy budget, we observed a clear drop in adversary advantage, plummeting to values even lower than -50. After further investigation, we saw the value of the adversary advantage was misleading because the \gls{tpr} was a lot higher then the other \gls{tpr}.

The high \gls{tpr} for lower privacy budgets can be attributed to the fact that, for an epsilon value of 0.5, data points were plotted significantly outside the original domain. Our empirical findings thus emphasize the necessity, also echoed in the literature, of addressing this issue. Both nD-Laplace and Piecewise mechanisms would likely exhibit these shortcomings in practical settings.

The two variants, grid-nD-Laplace and density-nD-Laplace, have demonstrated their ability to mitigate this issue in practice. Grid-nD-Laplace, in particular, showcased more stable results, with its privacy and utility metrics aligning closely with those of nD-Laplace. We introduced density-nD-Laplace to address the sensitivity of the nD-Laplace mechanism to data shapes. However, it fell short in practice, particularly with the Line and Circle datasets.

\section{\replaced{Limitations}{Constraints}}
\added{In this section, we discuss the limitations of the research that have been identified based on our observations of the results.}
\begin{enumerate}
    \item \textbf{Utility of \gls{optics}: }
\added{We experienced difficulties with the \gls{optics} clustering algorithm.
Our analysis revealed a higher number of clusters for \gls{optics} compared to K-Means and \gls{ag}. In addition, a substantial proportion of the data points were identified as noise. This indicates possible problems with the $minPts$ parameter (see Section section:cluster-utility-2). The algorithm is extremely sensitive to the uniform noise generated by the nD-Laplace mechanism for the same reason.
The Piecewise mechanism is better able to preserve specific data structures and, as a result, has a smaller impact on the performance degradation of the  \gls{optics} algorithm.}
    \item \textbf{3 > dimensions for shape datasets:}
\added{The majority of the research on data shapes has been conducted on 2-dimensional and 3-dimensional synthetic data. The research dimension for shapes could not be expanded as a result.
Due to the lack of visible feedback from plotting, these synthetic datasets are challenging to create. 
The scope of our research does not include the development of specific tooling for this, and was therefore omitted.
}
\item \textbf{Adversary advantage:} 
\added{The adversary advantage, while valuable, proved inadequate in capturing all crucial aspects of our privacy experiments. This limitation had been anticipated, as we had observed similar issues in the existing literature, which led us to incorporate the True Positive Rate (\gls{tpr}) as an additional metric. The varying reporting and utilization of metrics in related studies make it harder to draw broader conclusions. In future research, when existing literature highlights specific limitations, we should also define alternative metrics, preferably ones already established within the related work.}
\end{enumerate} 
\section{Future work}
The nD-Laplace mechanism, based on our findings, holds promise for clustering applications. While it showcases distinct advantages and disadvantages compared to the Piecewise mechanism, there is room for improvement, especially in handling diverse data forms. To further its practical applicability, we propose the following research directions:

\begin{enumerate}
\item \textbf{Examine impact of data shapes}: \added{The practicality of nD-Laplace relies on its ability to adapt to different data shapes and clustering algorithms. This work has brought attention to challenges posed by certain directed data shapes and sensitive clustering algorithms, such as \gls{optics}. Consequently, there are several approaches for future research aimed at mitigating the impact of nD-Laplace noise for specific use cases, as we have attempted to do with density-nD-Laplace.}. 
\item \textbf{Dimensionality in (synthetic) datasets}: Our current study limits the dimensionality to 7 and 9 for certain datasets. Expanding this to higher dimensions, especially for synthetic datasets, could provide deeper insights. 
\item \textbf{Incorporate categorical and binary data}: Enhancing nD-Laplace to support these data types, similar to the Piecewise mechanism, can broaden its practical applicability. 
\end{enumerate}

